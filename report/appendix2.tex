\chapter{Usability Test Script} \label{ap2:usabtest}

\section{Introduction}

This document intends to serve as a script to the realization of an usability test of an interface for a mobile application to be implemented for the \emph{Android} mobile operating system.
This application has features of a social network, and aims at sharing information about public transport networks, with that sharing component being performed by the users of the platform.

This test aims to perform the validation of low-level prototypes developed in order to provide an interface and an interaction flow to the mentioned application.

To the realization of this test, a sequential set of tasks should be performed by the users. It is intended to evaluate the easiness of interaction between the user and the proposed interface, with the purpose of detecting and solving possible problems before an implementation phase.

\section{Socio-demographic questions}

\subsection{Age}
\begin{itemize}
\item 18-25
\item 25-40
\item 40-60
\item 60+
\end{itemize}

\subsection{Gender}
\begin{itemize}
\item Male
\item Female
\end{itemize}

\subsection{How frequently do you use public transports?}
\begin{itemize}
\item Frequently (1-5 times per week).
\item Often (1-5 times per month).
\item Rarely (1-10 times per year).
\item Never.
\end{itemize}

\subsection{Are you a smartphone owner and user?}
\begin{itemize}
\item Yes.
\item No.
\end{itemize}

\subsection{Please state your city of residence.}

\section{List of Tasks to Accomplish}
\begin{itemize}

\item \textbf{Task 1 -} Add a New Feed Subscription

\begin{itemize}
\item Add a new feed, entering manually the origin and destination you desire. For the sake of this test, enter 'S. Bento' as the desired origin.
\item Add two of the displayed results to your current feeds (for instance, \textbf{S.Bento - Campanhã and S.Bento - General Torres}).
\end{itemize} 

\item \textbf{Task 2 -} Filter Visible Feeds

\begin{itemize}
\item Check the list of current subscribed feeds. Verify that those you added make part of that list.
\item Hide the information from the feed referring to '\textbf{S.Bento - General Torres}'.
\end{itemize} 

\item \textbf{Task 3 -} Remove Feed Subscription

\begin{itemize}
\item Remove one of the feeds (\textbf{Hospital S. João - Trindade}). Verify that it's no longer part of the list.
\end{itemize}

\item \textbf{Task 4 -} View Feed Subscriptions Information

\begin{itemize}
\item Read the information coming from the feeds subscribed.
\end{itemize}

\item \textbf{Task 5 -} Rate a Comment

\begin{itemize}
\item Rate another users' comment.
\end{itemize}

\item \textbf{Task 6 -} Submit a Comment

\begin{itemize}
\item Assume that you've just entered a bus from the route 604, with destination \textbf{Sto. Ovídio}. Submit a written comment about that trip.

\item Submit another comment about the trip, this time regarding the noise on the vehicle.
\end{itemize}

\item \textbf{Task 7 -} Check Favourites List

\begin{itemize}
\item Access your favourite journeys list. Assume that there are already two there (\textbf{Trindade - Sto. Ovídio} e \textbf{Trindade - Foz}). Deactivate the one that is currently active.
\end{itemize}

\item \textbf{Task 8 -} Check Scheduled Journeys

\begin{itemize}
\item Check your scheduled journeys. Add another journey to the same list.
\end{itemize}

\item \textbf{Task 9 -} User Profile

\begin{itemize}
\item Consult your user profile. Enter the list of reward to check the discounts you can claim with your current amount of points.
\end{itemize}

\end{itemize}


\section{Post-task questionnaire}

In a scale from 1 to 5, with 1 representing 'very hard' and 5 considered 'very easy', classify your perception of the difficulty of the task, based on the simplicity of the interaction and the clarity of messages displayed in the interface.

Finally (and optionally) give an answer to these questions:

\begin{itemize}
\item Would you use an application such as this in your daily routine?
\item What captivates you more about the application (one specific feature, the possible rewards, etc.=?
\end{itemize}