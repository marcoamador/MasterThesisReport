\chapter{Focus Group Script} \label{ap1:focgr}

\subsection{Goals}

This \emph{Focus Group} intends to present some uses cases from the application, in order to elicit usability requirements for a future functional prototype of the said application.
It is intended to promote a discussion which, recurring to suggestions of interface designs for some of the use cases, can serve as indication to the best way to proceed regarding the interaction of the application.

\subsection{Briefing to the participants}

A number increasingly higher of public transport passengers is connected everyday to social network platforms through the utilization of their personal mobile devices. This allows sharing of information between passengers in real time, regarding several aspects of the public transport service at a given moment.
This kind of information can be useful both to the users, in order to help them take more informed decisions regarding their journeys, and the transport network managers, helping them to introduce improvements to the service.
Regarding this last point, there can also be benefits to the public transport operators through the acquisition of operational information in a more timely manner and with less use of resources.

However, the most used social networks are incapable of aggregating and distributing custom information to passengers and managers of public transport networks, leading to the existence of spread information with much less utility. Therefore, it is intended to develop a prototype of a mobile service based on the structure of a social network, but innovative in the way how users are connected between each other, temporarily, based on their location and travel patterns. 

The developed prototype will be a new and more advanced version of other prototype previously developed, but with significant improvements in several aspects, such as usability and the capacity to establish temporary connections between passengers in an intelligent and real time manner.

\section{Use Cases to Discuss}

\subsection{Journey Planner and Check-In}

\begin{itemize}
\item Check-in must have better visibility than in the previous prototype. In that prototype, the journey planner, despite referring to a probable check-in in a future journey, had more visibility than a journey intended to perform in the immediate present.

\item It is impossible to verify is the user has several trips planned (it is not possible to list all the previously planned journeys), which brings up a difficulty if the user had planned more than one trip to perform in the future. The implementation of such feature has benefit?
\end{itemize}

\subsection{User Profile}

Displays information about the user profile in the application. Suggestions for data to display in this screen are:

\begin{itemize}

\item User Name
\item Photo/Avatar of the user.
\item Map with user location (and possibly users in the same network).
\item Information about vehicle or route where the user currently is.
\item Menu to access other features.
\item Access button to visibility/privacy options.

\end{itemize}

Regarding this use case, other questions that may be relevant of discussion:

\begin{itemize}
\item Is the user profile screen the best to display to the user when entering the application (instead of, for instance, travel information)?
\item Access to other features should be made from this screen or should the other feature be accessible on other 'tabs' from the application or through a swipe gesture?
\item Is this the right screen to display points and stars from the user?
\end{itemize}

\subsection{Trip information}

Regarding this use case, the following points must be considered:

\begin{itemize}
\item \textbf{Information displaying mode} - Should be implemented, for instance, under the form of a 'newsfeed' or there is a better way to do it? The task of 'rating' the information should be in another 'tab', to give visibility to that feature, or done recurring to buttons appearing in the news feed along with the information?

\item \textbf{Easiness of information submission} - The user must have minimum effort to submit new information. This means that the interface for quantitative rating of information should be as  clear as possible (with properly sized buttons or through sliders). The qualitative information submission (written comments) should also minimize text input effort from the user.

\item \textbf{Presentation of the types of information to submit} - Previous division by category (such as the previous prototype)? All types of information in the same page/screen (scrolling requires less effort than going back and choosing other category, despite the fact that the items in the end of the list become less visible)?
\end{itemize}

\section{Other questions to discuss regarding other features or interactions}

\begin{itemize}
\item \textbf{\emph{Early Registration}} - Eliminating early registration in the application, giving the user features that do not allow login or check-in in a network to access information could lead to less 'abandon rate' when first using the application.
However, it could lead to a vast number of users not creating an account to share information.

\item How relevant to the users is the possibility to see other users in their temporary network?

\item It would be interesting to provide users with the history of their trips and submitted comments?

\item \textbf{Notifications} - Sending a notification to the user when he enters a network or has a planned journey in the next 10 minutes is excessive?

\end{itemize}


