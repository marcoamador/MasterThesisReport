\chapter*{Abstract}

The advent of smartphones and the massification of Internet connections allowed easy access and sharing of real-time information between people. This fact has contributed to the success of several social networks and location-based services developed for mobile devices. The change of paradigm to accessing and sharing information may be explored in order to solve mobility problems, through improvements on public transport services and their use experience, promoting the use of public transport in urban areas.

Relating to this, a previous proposal was made in order to create a mobile application, with social network features, allowing real time sharing of public transport information between travellers, related to several aspects of the service. Such shared information can be useful to a better informed decision making by the users about their own journeys, and form a way that public transport operators can use in order to gather information about their service quality and, therefore, to improve it. Simultaneously, a good structure and organization of the shared information maximizes its utility to the users of the platform. That previous work resulted in a functional prototype with several limitations concerning the usability and user interaction with the application, conditioning the application's engagement and mass adherence by potential users.

This dissertation aims to develop a functional prototype of that application, applying user-centred design processes in order to produce a new iteration that overcomes the referred limitations, improving user experience and focusing in the interaction between the final user and the application. Initially, limitations of the previous iteration of the application were identified, along with some research in order to understand the usability requirements among potential users. After the prototype development (through an iterative process with evolution and validation of the designed interface and interaction), usability tests were performed among experts in the field of mobile applications development to public transport systems, which allowed to gather feedback about the work carried out and the potential range of an application with these characteristics.

This work can be seen as an evolution of the base concept behind this application, based upon user engagement in the development process phases, serving as a starting point to the integration of the addressed concepts in larger scale projects related to mobility.

\chapter*{Resumo}

O advento dos smartphones e a massificação das conexões à Internet possibilitou às pessoas o acesso e partilha de informação em tempo real com facilidade. Tal contribuiu para o sucesso de várias redes sociais e outros serviços baseados na localização do utilizador de dispositivos móveis. A alteração desse paradigma de acesso e partilha de informação poderá ser explorada no intuito de resolver problemas associados à mobilidade, através da melhoria dos serviços de transporte público e da sua experiência de utilização, promovendo a utilização desse tipo de transporte.

Nesse sentido, foi proposta anteriormente a criação de uma aplicação móvel com características de rede social com vista à troca de informações de transportes públicos entre os seus passageiros, em tempo real, sendo essas informações referentes a vários aspectos do serviço. A partilha desse tipo de informações poderá ser útil para uma tomada de decisão mais informada acerca das suas viagens, por parte dos utilizadores, e constituir uma forma através da qual os operadores de transporte público poderão recolher informação sobre o seu serviço com vista a melhorar a qualidade do mesmo. Simultaneamente, a estruturação e organização da informação partilhada maximiza a utilidade da mesma para os utilizadores da plataforma. Esse trabalho anterior resultou num protótipo funcional com limitações na óptica de usabilidade e interacção com o utilizador, o que condiciona uma adesão da mesma em larga escala por parte de potenciais utilizadores.

Esta dissertação tem como objectivo o desenvolvimento de um protótipo funcional para a dita aplicação, aplicando os processos de design centrado no utilizador de modo a produzir uma nova iteração dessa aplicação que ultrapasse essas limitações, melhorando a experiência de utilização e dando foco à interacção entre a aplicação e o utilizador final. Inicialmente, foi feito o levantamento das limitações da iteração anterior da aplicação móvel, tendo sido igualmente feita alguma pesquisa no sentido de perceber quais os requisitos de usabilidade de potenciais utilizadores. Após o desenvolvimento do referido protótipo (através de um processo iterativo com evolução e validação da interface e interacção concebidas), foram realizados testes de usabilidade, junto de experts na área do desenvolvimento de aplicações móveis para transportes públicos, o que permitiu recolher feedback acerca do trabalho realizado e do possível alcance de uma aplicação deste tipo. 

Este trabalho pode ser visto como uma evolução do conceito base desta aplicação, tendo por base o envolvimento dos utilizadores nas fases do processo de desenvolvimento, servindo de ponto de partida para a integração dos conceitos aqui abordados em projectos de maior escala ligados à mobilidade.
